%Connect FEniCS to QMCPy
\documentclass[11pt,compress,xcolor={usenames,dvipsnames},aspectratio=169]{beamer}
%\documentclass[xcolor={usenames,dvipsnames},aspectratio=169]{beamer} %slides and 
%notes
\usepackage[T1]{fontenc}
\usepackage{tgadventor} %Font found at https://tug.org/FontCatalogue/
%\usepackage{newpxtext}
\usepackage[euler-digits,euler-hat-accent]{eulervm}

\usepackage{amsmath,
	amssymb,
	datetime,
	mathtools,
	bbm,
	%mathabx,
	array,
	booktabs,
	xspace,
	multirow,
	calc,
	colortbl,
	siunitx,
 	graphicx}
\usepackage[usenames]{xcolor}
\usepackage[giveninits=false,backend=biber,style=nature, maxcitenames =10, mincitenames=9]{biblatex}
\addbibresource{FJHown23.bib}
\addbibresource{FJH23.bib}
\usepackage{media9}
\usepackage[autolinebreaks]{mcode}
\usepackage[tikz]{mdframed}


\usetheme{FJHSlimNoFoot169}
\setlength{\parskip}{2ex}
\setlength{\arraycolsep}{0.5ex}

\DeclareMathOperator{\GP}{\mathcal{G} \! \mathcal{P}}
\DeclareMathOperator{\SOL}{SOL}
\DeclareMathOperator{\APP}{APP}
\DeclareMathOperator{\ERR}{ERR}
\DeclareMathOperator{\AVG}{AVG}
\DeclareMathOperator{\INT}{INT}
\DeclareMathOperator{\LIN}{LINEAR}
\DeclareMathOperator{\BAD}{BAD}
%\DeclareMathOperator{\opt}{opt}
\newcommand{\dataN}{\bigl(\hf(\vk_i)\bigr)_{i=1}^n}
\newcommand{\dataNj}{\bigl(\hf(\vk_i)\bigr)_{i=1}^{n_j}}
\newcommand{\dataNjd}{\bigl(\hf(\vk_i)\bigr)_{i=1}^{n_{j^\dagger}}}
\newcommand{\ERRN}{\ERR\bigl(\dataN,n\bigr)}
\newcommand{\otod}{\ensuremath{1\mkern-4mu : \mkern-2mu d}}
\newcommand{\hvu}{\widehat{\vu}}
\newcommand{\hu}{\widehat{u}}


%\DeclareMathOperator{\app}{app}

\providecommand{\HickernellFJ}{H.\xspace}


\renewcommand{\OffTitleLength}{-7ex}
\setlength{\FJHThankYouMessageOffset}{-8ex}
\title{Connecting FEniCS to QMCPy}
\author[]{Fred Hickernell, hopefully with Mark Klinchin}
\institute{Department of Applied Mathematics \qquad
	Center for Interdisciplinary Scientific Computation \\
	Illinois Institute of Technology \qquad
	\href{mailto:hickernell@iit.edu}{\url{hickernell@iit.edu}} \qquad
	\href{http://mypages.iit.edu/~hickernell}{\url{mypages.iit.edu/~hickernell}}}

\thanksnote{}
	
%\event{Happy Fred}
\date[]{ revised \today}

\input FJHDef.tex



\begin{document}
	\everymath{\displaystyle}

\frame{\titlepage}


\begin{frame}{QMCPy $+$ FEniCS}

\vspace{-4ex}
\alert{Goal:} Demonstrate how to connect our package, QMCPy, with a major package that lacks what QMCPy provides

\vspace{2ex}

\uncover<2->{\begin{columns}[T]
\centering
\column{0.5\textwidth}
\centering
\alert{QMCPy} \href{https://qmcpy.org/}{\beamergotobutton{here}} 
\begin{itemize}
    \item Low discrepancy point generation, $\vw_1, \vw_2, \ldots$
    \item Automatic stopping criteria that determine the $n$ for approximating $\Ex[g(\vW)] = \int_{[0,1]^N} g(\vw) \, \dif \vw$, where $\vW \sim \cu[0,1]^N$, by $\frac 1n \sum_{i=1}^n g(\vw_i)$ with error tolerance $\varepsilon$
\end{itemize}

\column<3->{0.5\textwidth}{
\centering
\alert{FEniCS} \href{https://fenicsproject.org//}{\beamergotobutton{here}} 
\begin{itemize}
    \item State-of-the art finite element solver for partial differential equations (PDEs)
    \item Runs on high performance machines
    \item No \alert{randomness} in the PDEs
\end{itemize}
}
\end{columns}}
\end{frame}


\begin{frame}{A Model Problem}

\vspace{-4ex}

In this \href{https://qmcpy.org/2021/06/04/elliptic-pde-demo/}{\beamerbutton{QMCPy blog}} Pieterjan Robbe showed how to solve an ordinary differential equation with random coefficients
\begin{equation*}
-\frac{\dif }{\dif x}\biggl(a(x) \frac{\dif}{\dif x} u(x) \biggr) = f(x), \quad 0 \leq x \leq 1, \qquad
u(0) = u_0, \quad
u(1) = u_1
\end{equation*}
where $a$ is a \alert{random} function, and so $u$ is also a \alert{random} function.  The form of randomness assumed is $\log(a) \sim \GP(0,C)$, meaning that $\log(a(x_1)), \ldots \log(a(x_N))$ has a \alert{multivariate Gaussian (normal)} distribution with
\begin{itemize}
    \item Mean $\vzero$
    \item Covariance $\bigl( C(x_i,x_j) \bigr)_{i,j=1}^N$ (see Jupyter notebook for the definition of $C$)
\end{itemize}

Suppose that we want to find $\Ex[u(1/2;a)]$?

\end{frame}


\begin{frame}{Solving an ODE with Random Coefficients}

\vspace{-8ex}
\begin{gather*}
-\frac{\dif }{\dif x}\biggl(a(x) \frac{\dif}{\dif x} u(x) \biggr) = f(x), \quad 0 \leq x \leq 1, \qquad
u(0) = u_0, \quad
u(1) = u_1, \qquad \\
\log(a) \sim \GP(0,C), \qquad \Ex[u(1/2;a)] = ?
\end{gather*}

\vspace{-3ex}
\begin{itemize}
    \item Numerical solutions of ordinary differential equations generate $\hvu = \bigl(\hu(x_1;\vA), \ldots, \hu(x_N;\vA)\bigr)^T$, where $\vA = \bigl( a(x_1), \dots, a(x_N) \bigr)^T$; we will use FEniCS to generate $\hvu$
    \item  Define a distribution for $\vA$ via the transformation $\vA = \vPsi(\vW)$, where $\vW \sim \cu[0,1]^N$
    \item We want 
    \[
    \Ex[u(1/2,a)] \approx \Ex[\hu(1/2,\vA)] = \Ex[\hu(1/2,\vPsi(\vW))] \approx \frac 1n \sum_{i=1}^n \hu(1/2,\vPsi(\vw_i))
    \]
    where $\vw_1, \vw_2, \ldots$ are low discrepancy points
    
\end{itemize}

    
\end{frame}


\begin{frame}{Next Steps}

\vspace{-8ex}
\begin{gather*}
-\frac{\dif }{\dif x}\biggl(a(x) \frac{\dif}{\dif x} u(x) \biggr) = f(x), \quad 0 \leq x \leq 1, \qquad
u(0) = u_0, \quad
u(1) = u_1, \qquad \\
\log(a) \sim \GP(0,C), \qquad \Ex[u(1/2,a)] = ?
\end{gather*}

\vspace{-3ex}
\begin{enumerate}
    \item Look at FEniCS tutorials to learn how to generate $\hvu = \bigl(\hu(x_1;\vA), \ldots, \hu(x_N;\vA)\bigr)^T$, where $\vA = \bigl( a(x_1), \dots, a(x_N) \bigr)^T$
    \begin{itemize}
        \item First do $u_0 = u_1 = 0$, $a(x) = 1$, $f(x) = u(x) = \sin(\pi x)$
        \item Next try $a(x) = x$, $u(x) = \sin (\pi x)$, $f(x) =$ whatever it takes
    \end{itemize}
    \item  Implement the log-normal transformation, $\vPsi$ following what Pieterjan did.
    \item Use the Sobol' or lattice stopping criterion to approximate  $\Ex[\hu(1/2,\vPsi(\vW))]$, where $\vW \sim \cu[0,1]^N$
    
\end{enumerate}

    
\end{frame}




\end{document}